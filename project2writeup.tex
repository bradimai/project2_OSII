\documentclass[letterpaper,10pt,titlepage]{article}

\usepackage{graphicx}
\usepackage{amssymb}
\usepackage{amsmath}
\usepackage{amsthm}

\usepackage{alltt}
\usepackage{float}
\usepackage{color}
\usepackage{url}
\usepackage{listings}

\usepackage{balance}
\usepackage[TABBOTCAP, tight]{subfigure}
\usepackage{enumitem}
\usepackage{pstricks, pst-node}

\usepackage{geometry}
\geometry{textheight=8.5in, textwidth=6in}

%random comment

\newcommand{\cred}[1]{{\color{red}#1}}
\newcommand{\cblue}[1]{{\color{blue}#1}}

\usepackage{hyperref}
\usepackage{geometry}

\def\name{Bradley Imai and Daniel Ross}

%pull in the necessary preamble matter for pygments output
% \input{pygments.tex}

%% The following metadata will show up in the PDF properties
\hypersetup{
  colorlinks = true,
  urlcolor = black,
  pdfauthor = {\name},
  pdfkeywords = {CS444 ``operating systems'' files filesystem I/O},
  pdftitle = {CS 444 Project 2},
  pdfsubject = {CS 444 Project 2},
  pdfpagemode = UseNone
}

\begin{document}

\begin{titlepage}
    \begin{center}
        \vspace*{3.5cm}

        \textbf{Project 2}

        \vspace{0.5cm}

        \textbf{Bradley Imai and Daniel Ross}

        \vspace{0.8cm}

        CS 444\\
        Spring 2017\\
        5 May 2017\\

        \vspace{1cm}

        \textbf{Abstract}\\

        \vspace{0.5cm}

        abstract text\vfill


    \end{center}
\end{titlepage}

\newpage

\section{The design you plan to use to implement the SSTF algorithms}

text

\section{Concurrency Questions}

\textit{What do you think the main point of this assignment is?}\\

text\\

\textit{How did you personally approach the problem? Design decisions, algorithm, etc.}\\

text \\

\textit{How did you ensure your solution was correct? Testing details, for instance.}\\

text\\

\textit{What did you learn?}\\

text\\

\section{Version Control Log Github}
\begin{tabular}{lll} \textbf{Author}
     & \textbf{Date}
     & \textbf{Message}
\\ \hline
Bradimai & 2017-05-3 & Initial commit pushing starting files \\ \hline
RossDan96 & 2017-05-3 & Testing dining philosophers \\ \hline
RossDan96 & 2017-05-3 & made changes to the dining philosophers file \\ \hline
RossDan96 & 2017-05-3 & almost done with dining philosophers problem \\ \hline
Bradimai & 2017-05-3 & made a few changes to dining philosopher file\\ \hline
Bradimai & 2017-05-3 & took out print statements and cleaned up philosopher file \\ \hline
RossDan96 & 2017-05-3 &  pushed the final concurrency2 file\\ \hline
RossDan96 & 2017-05-3 & created a schedule and start project 2 \\ \hline

\end{tabular}

\section{Work Log}

\begin{tabular}{lll} \textbf{where}
     & \textbf{Date}
     & \textbf{what we did}

\\ \hline
OWen & 2017-05-2 & started the concurrency2 in recitation  \\ \hline
Linc & 2017-05-3 & stared implemented the code for the concurrency2 \\ \hline
Linc & 2017-05-3 &  later in the evening finished the concurrency 2\\ \hline
linc & 2017-05-3 & started reviewing are stated the project2 \\ \hline

\end{tabular}

My group member and I started our concurrency about a week ago or so. During our recitation we were able to get stated part of the concurrency2 assignment. We met up the following day at Linc to finish the concurrency2 assignment. We started the concurrency 2 solution a while back in one of our recitations so we had a pretty good start on how to implement the code to this problem. We met up in the morning to get a start on it and met up again in the evening to finish the concurrency2 problem. After finishing the concurrency2 assignment we started the project2. We download the appropriate files did some thinking on how we should tackle the problem. We met up again the following day to finish up the project2 part of the assignment.Finally we met up on Friday to finish up the written part of the assignment and a couple small changes to our code.

\end{document}
